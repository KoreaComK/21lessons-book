\addpart{Pensamentos Finais}
\pdfbookmark{Conclusion}{conclusion}
\label{ch:conclusion}

\chapter*{Conclusão}

\begin{chapquote}{Lewis Carroll, \textit{Alice no País das Maravilhas}}
\enquote{Comece do Começo} disse o Rei, seriamente, \enquote{continue 
até chegar a um fim: então pare.}
\end{chapquote}
Como mencionado no início, eu acho que qualquer resposta para a questão 
\textit{“O que você aprendeu com o Bitcoin?”} vai ser sempre incompleta. 
A simbiose de vários sistemas que podem ser vistos como vivos -- Bitcoin, 
a tecnoesfera, e economia -- é muito interligada, os tópicos muito numerosos, e 
as coisas estão se movendo muito rápido para uma pessoa sozinha entender tudo.

Mesmo sem entender tudo, e com todas suas especificidades e problemas, Bitcoin 
funciona, sem dúvida nenhuma. A cada dez minutos, um bloco é produzido, 
e o faz lindamente. Enquanto o Bitcoin continuar a trabalhar, mais pessoas 
vão entrar para usar.

\begin{quotation}\begin{samepage}
\enquote{É verdade que as coisas são lindas quando funcionam. Arte é função.}
\begin{flushright} -- Giannina Braschi\footnote{Giannina Braschi, \textit{Empire of Dreams} \cite{braschi2011empire}}
\end{flushright}\end{samepage}\end{quotation}

\paragraph{} Bitcoin é uma criança da internet. Cresce exponencialmente, 
apagando as divisões entre as disciplinas. Não é claro, por exemplo, onde o 
reino da tecnologia pura acaba e onde outro reino começa. Mesmo que o Bitcoin 
precise de computadores para funcionar eficientemente, ciências da computação não 
é o suficiente para entender. Bitcoin não é só sem fronteiras no seu funcionamento, 
mas também no respeito de suas disciplinas acadêmicas.

Economia, política, teoria de jogos, história monetária, teoria de redes, finança, 
criptografia, teoria da informação, censura, leis e regulação, organização humana, 
psicologia -- todas essas áreas de aprendizado que podem ajudar na aventura de 
entender como o Bitcoin funciona e o que o Bitcoin é.

Nenhuma invenção sozinha é responsável pelo seu sucesso. É uma combinação, de 
várias peças que não tinham relação entre si, coladas pela teoria de jogos e incentivos, 
que fazem o Bitcoin ser revolucionário como é. Uma bela mistura de muitas disciplinas 
é o que faz o Satoshi um gênio.

\paragraph{} Como todo sistema complexo, Bitcoin tem que fazer trocas 
em termos de eficiência, custo, segurança e muitas outras propriedades. Assim 
como tentar encaixar um cubo em um círculo, não existe uma solução perfeita, 
toda solução para os problemas que o Bitcoin tentar resolver vai ser sempre imperfeita.

\begin{quotation}\begin{samepage}
\enquote{ Eu não acredito que teremos um bom dinheiro novamente, antes de 
	tirarmos ele das mãos do governo, embora não possamos tirá-lo violentamente 
	das mãos do governo, tudo que podemos fazer é contorná-lo de algum modo, 
	introduzir algo que eles não possam parar.}
\begin{flushright} -- Friedrich Hayek\footnote{Friedrich Hayek on Monetary Policy, the Gold Standard, Deficits, Inflation, and John Maynard Keynes \url{https://youtu.be/EYhEDxFwFRU}}
\end{flushright}\end{samepage}\end{quotation}

Bitcoin é a maneira marota de re-introduzir um bom dinheiro para o mundo. 
Ele faz isso colocando um indivíduo soberano atrás de cada nó, assim como Leonardo Da Vinci 
tentou resolver o problema intratável de colocar um quadrado dentro de um círculo, 
colocando o Homem Vitruviano no seu centro. Nós efetivamente removem qualquer conceito de centro, 
criando um sistema que é imensamente antifrágil e extremamente difícil de desligar. 
Bitcoin Vive, e seu batimento provavelmente irá durar mais do que os nossos.

Eu espero que você tenha gostado dessas vinte e uma lições. Talvez a lição mais 
importante é que o Bitcoin deve ser examinado de maneira Holística, de vários ângulos, 
se você quisesse ter alguma coisa que se aproxime de uma visão completa. 
Assim como remover uma parte de um sistema complexo destrói e desregula o todo, 
examinar partes do Bitcoin isoladamente parece que prejudica o nosso entendimento. 
Se apenas uma pessoa tirar \enquote{blockchain} do seu vocabulário, e trocar por 
\enquote{uma corrente de blocos} Irei morrer como um homem feliz.

De qualquer jeito, minha jornada continua. Eu planejo me aventurar mais 
profundamente na toca do coelho, e te convido para caminhar ao meu lado pela jornada.
\footnote{\url{https://twitter.com/dergigi}}

% <!-- Twitter -->
% [dergigi]: https://twitter.com/dergigi
%
% <!-- Internal -->
% [sly roundabout way]: https://youtu.be/EYhEDxFwFRU?t=1124
% [Giannina Braschi]: https://en.wikipedia.org/wiki/Braschi%27s_Empire_of_Dreams
