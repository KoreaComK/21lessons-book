\addpart{Final Thoughts}
\pdfbookmark{Conclusion}{conclusion}
\label{ch:conclusion}

\chapter*{Conclusion}

\begin{chapquote}{Lewis Carroll, \textit{Alice in Wonderland}}
\enquote{Begin at the beginning} the King said, very gravely, \enquote{and go on till you
come to the end: then stop.}
\end{chapquote}

As mentioned in the beginning, I think that any answer to the
question \textit{“What have you learned from Bitcoin?”} will always be incomplete. The
symbiosis of what can be seen as multiple living systems -- Bitcoin, the
technosphere, and economics -- is too intertwined, the topics too numerous, and
things are moving too fast to ever be fully understood by a single person.

Even without understanding it fully, and even with all its quirks and seeming
shortcomings, Bitcoin undoubtedly works. It keeps producing blocks roughly every
ten minutes and does so beautifully. The longer Bitcoin continues to work, the
more people will opt-in to use it.

\begin{quotation}\begin{samepage}
\enquote{It's true that things are beautiful when they work. Art is function.}
\begin{flushright} -- Giannina Braschi\footnote{Giannina Braschi, \textit{Empire of Dreams} \cite{braschi2011empire}}
\end{flushright}\end{samepage}\end{quotation}

\paragraph{} Bitcoin is a child of the internet. It is growing exponentially,
blurring the lines between disciplines. It isn’t clear, for example, where the
realm of pure technology ends and where another realm begins. Even though
Bitcoin requires computers to function efficiently, computer science is not
sufficient to understand it. Bitcoin is not only borderless in regards to its
inner workings but also boundaryless in respect to academic disciplines.

Economics, politics, game theory, monetary history, network theory, finance,
cryptography, information theory, censorship, law and regulation, human
organization, psychology -- all these and more are areas of expertise which might
help in the quest of understanding how Bitcoin works and what Bitcoin is.

No single invention is responsible for its success. It is the combination of
multiple, previously unrelated pieces, glued together by game theoretical
incentives, which make up the revolution that is Bitcoin. The beautiful blend of
many disciplines is what makes Satoshi a genius.

\paragraph{} Like every complex system, Bitcoin has to make tradeoffs in terms
of efficiency, cost, security, and many other properties. Just like there is no
perfect solution to deriving a square from a circle, any solution to the
problems which Bitcoin tries to solve will always be imperfect as well.

\begin{quotation}\begin{samepage}
\enquote{I don’t believe we shall ever have a good money again before we take the
thing out of the hands of government, that is, we can’t take it violently
out of the hands of government, all we can do is by some sly roundabout way
introduce something that they can’t stop.}
\begin{flushright} -- Friedrich Hayek\footnote{Friedrich Hayek on Monetary Policy, the Gold Standard, Deficits, Inflation, and John Maynard Keynes \url{https://youtu.be/EYhEDxFwFRU}}
\end{flushright}\end{samepage}\end{quotation}

Bitcoin is the sly, roundabout way to re-introduce good money to the world. It
does so by placing a sovereign individual behind each node, just like Da Vinci
tried to solve the intractable problem of squaring a circle by placing the
Vitruvian Man in its center. Nodes effectively remove any concept of a center,
creating a system which is astonishingly antifragile and extremely hard to shut
down. Bitcoin lives, and its heartbeat will probably outlast all of ours.

I hope you have enjoyed these twenty-one lessons. Maybe the most important
lesson is that Bitcoin should be examined holistically, from multiple angles, if
one would like to have something approximating a complete picture. Just like
removing one part from a complex system destroys the whole, examining parts of
Bitcoin in isolation seems to taint the understanding of it. If only one person
strikes \enquote{blockchain} from her vocabulary and replaces it with \enquote{a
chain of blocks} I will die a happy man.

In any case, my journey continues. I plan to venture further down into the
depths of this rabbit hole, and I invite you to tag
along for the ride.\footnote{\url{https://twitter.com/dergigi}}

% <!-- Twitter -->
% [dergigi]: https://twitter.com/dergigi
%
% <!-- Internal -->
% [sly roundabout way]: https://youtu.be/EYhEDxFwFRU?t=1124
% [Giannina Braschi]: https://en.wikipedia.org/wiki/Braschi%27s_Empire_of_Dreams
