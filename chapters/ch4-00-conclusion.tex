\addpart{Considerações Finais}

\pdfbookmark{Conclusão}{Conclusão}
\label{ch:conclusion}

\chapter*{Conclusão}


\begin{chapquote}{Lewis Carroll, \textit{Alice no País das Maravilhas}}
\enquote{Comece pelo começo}, disse o Rei, gravemente, \enquote{e prossiga até chegar ao fim; então pare}.
\end{chapquote}

Como mencionado no início, eu acho que qualquer resposta para a questão 
\textit{“O que você aprendeu com o Bitcoin?”} será sempre incompleta. 
A simbiose de vários sistemas que podem ser vistos como vivos -- Bitcoin, 
a tecnosfera, e economia -- é muito interligada, os tópicos muito numerosos, e 
as coisas estão se movendo muito rápido para uma pessoa sozinha entender tudo.

Mesmo sem entendê-lo totalmente, e mesmo com todas as suas peculiaridades e aparentes deficiências, o Bitcoin, sem dúvida, funciona. Ele continua produzindo blocos aproximadamente a cada dez minutos e faz isso lindamente. Quanto mais tempo o Bitcoin continua funcionando, mais pessoas optam por usá-lo.

\begin{quotation}\begin{samepage}
\enquote{É verdade que as coisas são bonitas quando funcionam. Arte é funcionamento.}
\begin{flushright} -- Giannina Braschi\footnote{Giannina Braschi, \textit{O Empério dos Sonhos} \cite{braschi2011empire}}
\end{flushright}\end{samepage}\end{quotation}

\paragraph{}
O Bitcoin é um filho da Internet. Está crescendo exponencialmente, confundindo as linhas entre as disciplinas. Não está claro, por exemplo, onde termina o reino da tecnologia pura e onde começa outra matéria. Embora o Bitcoin exija que os computadores funcionem com eficiência, a ciência da computação não é suficiente para entendê-lo. O Bitcoin não é apenas sem fronteiras no que diz respeito ao seu funcionamento interno, mas também sem fronteiras no que diz respeito a disciplinas acadêmicas.

Economia, política, teoria dos jogos, história monetária, teoria das redes, finanças, criptografia, teoria da informação, censura, lei e regulamentação, organização humana, psicologia --- tudo isso e mais são áreas de conhecimento que podem ajudar na busca pelo conhecimento de como o Bitcoin funciona e o que ele é.

Nenhuma invenção é responsável por seu sucesso. É a combinação de várias peças, que antes não se relacionavam, unidas por incentivos da teoria dos jogos, que constituem a revolução que é o Bitcoin. A bela mistura de muitas disciplinas é o que torna Satoshi um gênio.

\paragraph{}
Como todo sistema complexo, o Bitcoin precisa fazer concessões em termos de eficiência, custo, segurança e muitas outras propriedades. Assim como não existe uma solução perfeita para derivar um quadrado de um círculo, qualquer solução para os problemas que o Bitcoin tenta resolver sempre será imperfeita também.

\begin{quotation}\begin{samepage}
\enquote{Não acredito que algum dia teremos um bom dinheiro novamente antes de tirarmos isso das mãos do governo, ou seja, não podemos tirá-lo violentamente das mãos do Estado, tudo o que podemos fazer é por alguma astuta forma indireta, introduzir algo que eles não podem parar.}
\begin{flushright} -- Friedrich Hayek\footnote{Friedrich Hayek sobre Política Monetária, o Padrão Ouro, Déficits, Inflação, e John Maynard Keynes \url{https://youtu.be/EYhEDxFwFRU}}
\end{flushright}\end{samepage}\end{quotation}

O bitcoin é a maneira astuta e indireta de reintroduzir um bom dinheiro ao mundo. Ele faz isso colocando um indivíduo soberano atrás de cada node, assim como Da Vinci tentou resolver o problema intratável de transformar um quadrado em um círculo, colocando o Homem Vitruviano em seu centro. Os nodes removem efetivamente qualquer conceito de centralidade, criando um sistema surpreendentemente antifrágil e extremamente difícil de ser desligado. O Bitcoin vive e seu batimento cardíaco provavelmente durará mais que o nosso.

Espero que você tenha gostado dessas vinte e uma lições. Talvez a lição mais importante seja que o Bitcoin deve ser examinado holisticamente de vários ângulos, se alguém quiser ter algo próximo de uma imagem completa. Assim como remover uma parte de um sistema complexo destrói o todo, examinar partes do Bitcoin isoladamente parece contaminar a compreensão dele. Se apenas uma pessoa eliminar o termo \enquote{blockchain} de seu vocabulário e substituí-lo por \enquote{cadeia de blocos}, morrerei feliz.

Em qualquer caso, minha jornada continua. Pretendo me aventurar ainda mais nas profundezas desta toca do coelho e convido você a vir junto no passeio.\footnote{\url{https://twitter.com/dergigi}}

% <!-- Twitter -->
% [dergigi]: https://twitter.com/dergigi
%
% <!-- Internal -->
% [sly roundabout way]: https://youtu.be/EYhEDxFwFRU?t=1124
% [Giannina Braschi]: https://en.wikipedia.org/wiki/Braschi%27s_Empire_of_Dreams
