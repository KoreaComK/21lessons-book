\chapter{Introduction}
\label{ch:introduction}

\newthought{In October 2018}, Arjun Balaji asked the innocuous question,
\textit{What have you learned from Bitcoin?}

After trying to answer this question in a short tweet, and failing miserably, I
realized that the things I've learned are far too numerous to answer quickly, if
at all.

The things I've learned are, obviously, about Bitcoin - or at least related to
it. However, while some of the inner workings of Bitcoin are explained, the
following lessons are not an explanation of how Bitcoin works or what it is,
they might, however, help to explore some of the things Bitcoin touches:
philosophical questions, economic realities, and technological innovations.

The 21 lessons are structured in bundles of seven, resulting in three chapters.
Each chapter looks at Bitcoin through a different lens, extracting what
lessons can be learned by inspecting this strange network from a different
angle.

\hyperref[ch:philosophy]{Chapter 1} explores the philosophical teachings of
Bitcoin. The interplay of immutability and change, the concept of true scarcity,
Bitcoin's immaculate conception, the problem of identity, the contradiction of
replication and locality, the power of free speech, and the limits of knowledge.

\hyperref[ch:economics]{Chapter 2} explores the economic teachings of Bitcoin.
Lessons about financial ignorance, inflation, value, money and the history of
money, fractional reserve banking, and how Bitcoin is re-introducing sound money
in a sly, roundabout way.

\hyperref[ch:technology]{Chapter 3} explores some of the lessons learned by
examining the technology of Bitcoin.  Why there is strength in numbers,
reflections on trust, why telling time takes work, how moving slowly and not
breaking things is a feature and not a bug, what Bitcoin's creation can tell us
about privacy, why cypherpunks write code (and not laws), and what metaphors
might be useful to explore Bitcoin's future.

Each lesson contains several quotes and links throughout the text. If I have
explored an idea in more detail, you can find links to my related works in the
``Through the Looking-Glass'' section. If you like to go deeper, links to the most
relevant material are listed in the ``Down the Rabbit Hole'' section. Both can be
found at the end of each lesson.

Even though some prior knowledge about Bitcoin is beneficial, I hope that these
lessons can be digested by any curious reader. While some relate to each other,
each lesson should be able to stand on its own and can be read independently. I
did my best to shy away from technical jargon, even though some domain-specific
vocabulary is unavoidable.

I hope that my writing serves as inspiration for others to dig beneath the
surface and examine some of the deeper questions Bitcoin raises. My own
inspiration came from a multitude of authors and content creators to all of whom
I am eternally grateful.

Last but not least: my goal in writing this is not to convince you of anything.
My goal is to make you think, and show you that there is way more to Bitcoin
than meets the eye. I can’t even tell you what Bitcoin is or what Bitcoin will
teach you. You will have to find that out for yourself.

% > "After this, there is no turning back. You take the blue pill --- the
% > story ends, you wake up in your bed and believe whatever you want to
% > believe. You take the red pill --- you stay in Wonderland, and I show
% > you how deep the rabbit hole goes."
% > <cite>[Morpheus][Morpheus]</cite>

% 
%
% [Morpheus]: https://en.wikipedia.org/wiki/Red_pill_and_blue_pill#The_Matrix_(1999)
% [this question]: https://twitter.com/arjunblj/status/1050073234719293440
%
% <!-- Internal -->
% [chapter1]: {{ 'bitcoin/lessons/ch1-00-philosophy' | absolute_url }}
% [chapter2]: {{ 'bitcoin/lessons/ch2-00-economics' | absolute_url }}
% [chapter3]: {{ 'bitcoin/lessons/ch3-00-technology' | absolute_url }}
%
% <!-- Wikipedia -->
% [alice]: https://en.wikipedia.org/wiki/Alice%27s_Adventures_in_Wonderland
% [carroll]: https://en.wikipedia.org/wiki/Lewis_Carroll
