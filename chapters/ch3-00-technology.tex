\part{Tecnologia}
\label{ch:technology}
\chapter*{Tecnologia}

\begin{chapquote}{Lewis Carroll, \textit{Alice no País das Maravilhas}}
\enquote{Desta vez vou me sair melhor}, disse para si mesma, e começou por pegar a chavezinha de ouro e destrancar a porta que dava para o jardim.
\end{chapquote}

Chaves de ouro, relógios que só funcionam por acaso, corridas para resolver enigmas estranhos e construtores que não têm rostos nem nomes. O que parecem histórias de faz de conta é comum no mundo do Bitcoin.

Como exploramos no Capítulo~\ref{ch:economics}, grandes partes do sistema financeiro atual estão sistematicamente quebradas. Como Alice, só podemos esperar administrar melhor desta vez. Mas graças a um inventor pseudoanônimo temos uma tecnologia incrivelmente sofisticada para nos apoiar desta vez: o Bitcoin.

Resolver problemas em um ambiente radicalmente descentralizado e adversário requer soluções únicas. O que de outra forma seriam problemas triviais para resolver são tudo menos isso neste mundo estranho. O Bitcoin depende de uma criptografia forte para a maioria das soluções, pelo menos quando analisado através das lentes da tecnologia. Iremos explorar o quão forte essa criptografia é em uma das lições a seguir.

A criptografia é o que o Bitcoin usa para remover a confiança em autoridades. Ao invés de depender de instituições centralizadas, o sistema depende da autoridade final do nosso universo: a física. Alguns pequenos grãos de confiança ainda são necessários, no entanto. Examinaremos isso na segunda lição deste capítulo.

~

\begin{samepage}
Parte~\ref{ch:technology} -- Tecnologia:

\begin{enumerate}
  \setcounter{enumi}{14}
  \item O poder dos números
  \item Reflexões sobre: \enquote{Não Confie, Verifique!}
  \item Tempo demanda trabalho
  \item Mova-se lentamente e não quebre as coisas
  \item A Privacidade não morreu
  \item Cypherpunks escrevem códigos
  \item Metáforas para o futuro do Bitcoin
\end{enumerate}
\end{samepage}

As últimas duas lições exploram o \textit{ethos} do desenvolvimento tecnológico no Bitcoin, que é indiscutivelmente tão importante quanto a própria tecnologia. O Bitcoin não é o próximo aplicativo revolucionário no seu celular. É a base de uma nova realidade econômica, razão pela qual o Bitcoin deve ser tratado como um software financeiro de nível nuclear.

Onde estamos nesta revolução financeira, social e tecnológica? Redes e tecnologias do passado podem servir como metáforas para o futuro do Bitcoin, que são exploradas na última lição deste capítulo.

Mais uma vez, aperte o cinto e aproveite o passeio. Como todas as tecnologias exponenciais, estamos prestes a nos tornar parabólicos.