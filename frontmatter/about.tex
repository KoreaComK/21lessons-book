
\def\bitcoinB{\leavevmode
  {\setbox0=\hbox{\textsf{B}}%
    \dimen0\ht0 \advance\dimen0 0.2ex
    \ooalign{\hfil \box0\hfil\cr
      \hfil\vrule height \dimen0 depth.2ex\hfil\cr
    }%
  }%
}

\chapter*{Sobre Este Livro \\ (... e Sobre o Autor)}
\pdfbookmark{About This Book (... and About the Author)}{about}

Este é um livro um tanto incomum. Mas, o Bitcoin é uma tecnologia um tanto quanto incomum, então um livro incomum sobre o Bitcoin pode ser adequado. Não tenho certeza se sou um cara incomum (gosto de me ver como um cara \textit {normal}), mas a história de como esse livro surgiu e como me tornei um autor vale a pena ser dita.

Em primeiro lugar, não sou um escritor. Eu sou um engenheiro. Não estudei redação. Estudei código e programação. Em segundo lugar, nunca tive a intenção de escrever um livro, muito menos um livro sobre Bitcoin. Inferno! Eu nem falo o inglês nativo. \footnote{A razão pela qual estou escrevendo essas palavras em inglês é que meu cérebro funciona de maneiras misteriosas. Sempre que surge algo técnico, ele muda para o modo inglês.} Eu sou apenas um cara pego pela febre do Bitcoin. 

Quem sou \textit {eu} para escrever um livro sobre Bitcoin? Esta é uma boa pergunta. A resposta curta é simples: Sou Gigi, um bitcoinheiro.

A resposta longa é um pouco mais complicada.

\paragraph{}
Minha formação é em ciência da computação e desenvolvimento de software. Anteriormente, fiz parte de um grupo de pesquisa que tentava fazer os computadores pensarem e raciocinarem, entre outras coisas. Antes do Bitcoin, escrevi um software para processamento automatizado de passaportes e coisas relacionadas que ainda são bem assustadores. Eu sei uma ou duas coisas sobre computadores e nosso mundo interligado, então acho que tenho um pouco de vantagem para entender o lado técnico do Bitcoin. No entanto, como tento mostrar nesse livro, o lado técnico das coisas é apenas uma pequena fatia da Quimera que é o Bitcoin. E cada um destes pequenos pedaços são importantes.

Este livro surgiu por causa de uma pergunta simples: {\enquote{O que você aprendeu com o Bitcoin?}} Tentei responder a essa pergunta em um único tweet. Então o tweet se transformou em uma thread. A tempestade de tweets se transformou em um artigo. O artigo se transformou em três artigos. Três artigos se transformaram em 21 lições. E 21 lições se transformaram neste livro. Acho que sou péssimo em condensar meus pensamentos em um único tweet.

\paragraph{}
\textit{\enquote{Por que escreveu este livro?}}, você pode me perguntar. Novamente, há uma resposta curta e uma longa. A resposta curta é que eu simplesmente tinha que fazê-lo. Eu era (e ainda sou) {possuído} pelo Bitcoin. Acho que ele é infinitamente fascinante. Não consigo parar de pensar nisso e nas implicações que terá em nossa sociedade global. A resposta longa é que acredito que o Bitcoin é a invenção mais importante do nosso tempo, e mais pessoas precisam entender a natureza dessa invenção. O Bitcoin ainda é um dos fenômenos mais incompreendidos de nosso mundo moderno, e levei anos para perceber em toda sua completude a gravidade dessa tecnologia alienígena. Perceber o que é o Bitcoin e como ele transformará nossa sociedade é uma experiência profunda. Espero plantar as sementes que podem levar a essa compreensão em sua cabeça.

Embora esta seção seja intitulada \enquote{\textit{Sobre este livro (... e sobre o autor)}}, no frigir dos ovos, este livro, quem sou e o que fiz, realmente não importa. Eu sou apenas um node na rede, literalmente \textit{e} figurativamente. Além disso, não deve confiar no que estou dizendo, de maneira nenhuma. Como node e bitcoinheiro, gosto de dizer: faça sua própria pesquisa e o mais importante: não confie, verifique.

Fiz o meu melhor para fazer meu dever de casa e fornecer muitas fontes para você poder mergulhar de cabeça, caro leitor. Além das notas de rodapé e citações neste livro, tento manter uma lista atualizada de recursos em \href{https://21lessons.com/rabbithole}{21lessons.com/rabbithole} e \href{https://bitcoin-resources.com}{bitcoin-resources.com}, que também lista muitos outros conteúdos, livros e podcasts seleciodados que o ajudarão a entender o que é o Bitcoin.

\paragraph{}
Resumindo, este é um simples livro sobre o Bitcoin, escrito por um bitconheiro. O Bitcoin não precisa deste livro e você provavelmente não precisa deste livro para entender o Bitcoin. Acredito que o Bitcoin será compreendido por você assim que {você} estiver pronto, e também acredito que as primeiras frações de um bitcoin o encontrarão assim que você estiver pronto para recebê-las. Em essência, todos terão \bitcoinB{}itcoin no momento certo. Enquanto isso, o Bitcoin simplesmente é, e isso é o suficiente. \footnote{Beautyon, \textit{Bitcoin is. And that is enough.}~\cite{bitcoin-is}}