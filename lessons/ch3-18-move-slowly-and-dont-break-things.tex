\chapter{Lesson 18: Move Slowly and Don't Break Things}
\label{les:18}

\begin{chapquote}{Lewis Carroll, \textit{Alice in Wonderland}}
So the boat wound slowly along, beneath the bright summer-day, with its merry crew and its music of voices and laughter...
\end{chapquote}

It might be a dead mantra, but "move fast and break things" is still how
much of the tech world operates. The idea that it doesn't matter if you
get things right the first time is a basic pillar of the *fail early,
fail often* mentality. Success is measured in growth, so as long as you
are growing everything is fine. If something doesn't work at first you
simply pivot and iterate. In other words: throw enough shit against the
wall and see what sticks.

Bitcoin is very different. It is different by design. It is different
out of necessity. As Satoshi [pointed out], e-currency has been tried
many times before, and all previous attempts have failed because there
was a head which could be cut off. The novelty of Bitcoin is that it is
a beast without heads.

\begin{quotation}
``A lot of people automatically dismiss e-currency as a lost cause
because of all the companies that failed since the 1990's. I hope it's
obvious it was only the centrally controlled nature of those systems
that doomed them.''
\end{quotation}
% > <cite>[Satoshi Nakamoto][pointed out]</cite>

One consequence of this radical decentralization is an inherent
resistance to change. "Move fast and break things" does not and will
never work on the Bitcoin base layer. Even if it would be desirable, it
wouldn't be possible without convincing *everyone* to change their ways.
That's distributed consensus. That's the nature of Bitcoin.

\begin{quotation}
``The nature of Bitcoin is such that once version 0.1 was released, the
core design was set in stone for the rest of its lifetime.''
\end{quotation}
% > <cite>[Satoshi Nakamoto][4]</cite>

This is one of the many paradoxical properties of Bitcoin. We all came
to believe that anything which is software can be changed easily. But
the nature of the beast makes changing it bloody hard.

As Hasu beautifully shows in [Unpacking Bitcoin's Social Contract],
changing the rules of Bitcoin is only possible by *proposing* a change,
and consequently *convincing* all users of Bitcoin to adopt this change.
This makes Bitcoin very resilient to change, even though it is software.

This resilience is one of the most important properties of Bitcoin.
Critical software systems have to be antifragile, which is what the
interplay of Bitcoin's social layer and its technical layer guarantees.
Monetary systems are adversarial by nature, and as we have known for
thousands of years solid foundations are essential in an adversarial
environment.

\begin{quotation}
``The rain came down, the floods came, and the winds blew, and beat on
that house; and it didn't fall, for it was founded on the rock.''
\end{quotation}
% > <cite>[Matthew 7:24--27]</cite>

Arguably, in this parable of the wise and the foolish builders Bitcoin
isn't the house. It is the rock. Unchangeable, unmoving, providing the
foundation for a new financial system.

Just like geologists, who know that rock formations are always moving
and evolving, one can see that Bitcoin is always moving and evolving as
well. You just have to know where to look and how to look at it.

The introduction of [pay to script hash] and [segregated witness] are
proof that Bitcoin's rules can be changed if enough users are convinced
that adopting said change is to the benefit of the network. The latter
enabled the development of the [lightning network], which is one of the
houses being built on Bitcoin's solid foundation. Future upgrades like
[Schnorr signatures] will enhance efficiency and privacy, as well as
scripts (read: smart contracts) which will be indistinguishable from
regular transactions thanks to [Taproot]. Wise builders do indeed build
on solid foundations.

Satoshi wasn't only a wise builder technologically. He also understood
that it would be necessary to make wise decisions ideologically.

\begin{quotation}
``Being open source means anyone can independently review the code. If
it was closed source, nobody could verify the security. I think it's
essential for a program of this nature to be open source.''
\end{quotation}
% > <cite>[Satoshi Nakamoto][5]</cite>

Openness is paramount to security and inherent in open source and the
free software movement. As Satoshi pointed out, secure protocols and the
code which implements them have to be open --- there is no security
through obscurity. Another benefit is again related to decentralization:
code which can be run, studied, modified, copied, and distributed freely
ensures that it is spread far and wide.

The radically decentralized nature of Bitcoin is what makes it move
slowly and deliberately. A network of nodes, each run by a sovereign
individual, is inherently resistant to change --- malicious or not. With
no way to force updates upon users the only way to introduce changes is
by slowly convincing each and every one of those individuals to adopt a
change. This non-central process of introducing and deploying changes is
what makes the network incredibly resilient to malicious changes. It is
also what makes fixing broken things more difficult than in a
centralized environment, which is why everyone tries not to break things
in the first place.

Bitcoin taught me that moving slowly is one of its features, not a bug.

% ---
%
% #### Through the Looking-Glass
%
% - [Lesson 1: Immutability and Change][lesson1]
%
% #### Down the Rabbit Hole
%
% - [Unpacking Bitcoin's Social Contract] by Hasu
% - [Schnorr signatures BIP][Schnorr signatures] by Pieter Wuille
% - [Taproot proposal][Taproot] by Gregory Maxwell
% - [P2SH][pay to script hash], [SegWit][segregated witness] on the Bitcoin Wiki
% - [Parable of the Wise and the Foolish Builders][Matthew 7:24--27] on Wikipedia
%
% <!-- Down the Rabbit Hole -->
% [lesson1]: {{ '/bitcoin/lessons/ch1-01-immutability-and-change' | absolute_url }}
%
% [pointed out]: http://p2pfoundation.ning.com/forum/topics/bitcoin-open-source?commentId=2003008%3AComment%3A9493
% [4]: https://bitcointalk.org/index.php?topic=195.msg1611#msg1611
% [Unpacking Bitcoin's Social Contract]: https://uncommoncore.co/unpacking-bitcoins-social-contract/
% [Matthew 7:24--27]: https://en.wikipedia.org/wiki/Parable_of_the_Wise_and_the_Foolish_Builders
% [pay to script hash]: https://en.bitcoin.it/wiki/Pay_to_script_hash
% [segregated witness]: https://en.bitcoin.it/wiki/Segregated_Witness
% [lightning network]: https://lightning.network/
% [Schnorr signatures]: https://github.com/sipa/bips/blob/bip-schnorr/bip-schnorr.mediawiki#cite_ref-6-0
% [Taproot]: https://lists.linuxfoundation.org/pipermail/bitcoin-dev/2018-January/015614.html
% [5]: https://bitcointalk.org/index.php?topic=13.msg46#msg46
%
% <!-- Wikipedia -->
% [alice]: https://en.wikipedia.org/wiki/Alice%27s_Adventures_in_Wonderland
% [carroll]: https://en.wikipedia.org/wiki/Lewis_Carroll
