\chapter{Financial Ignorance}
\label{les:8}

\begin{chapquote}{Lewis Carroll, \textit{Alice in Wonderland}}
\enquote{And what an ignorant little girl she'll think me for asking! No, it'll never
do to ask: perhaps I shall see it written up somewhere.}
\end{chapquote}

One of the most surprising things, to me, was the amount of finance,
economics, and psychology required to get a grasp of what at first
glance seems to be a purely \textit{technical} system --- a computer network.
To paraphrase a little guy with hairy feet: \enquote{It's a dangerous business,
Frodo, stepping into Bitcoin. You read the whitepaper, and if you don't
keep your feet, there's no knowing where you might be swept off to.}

To understand a new monetary system, you have to get acquainted with the
old one. I began to realize very soon that the amount of financial
education I enjoyed in the educational system was essentially \textit{zero}.

Like a five-year-old, I began to ask myself a lot of questions: How does the
banking system work? How does the stock market work? What is fiat money? What is
\textit{regular} money? Why is there so much
debt?\footnote{\url{https://www.usdebtclock.org/}} How much money is actually
printed, and who decides that?

After a mild panic about the sheer scope of my ignorance, I found
reassurance in realizing that I was in good company.

\begin{samepage}\begin{quotation}
\enquote{Isn't it ironic that Bitcoin has taught me more about money than all these
years I've spent working for financial institutions? \ldots including starting my
career at a central bank}
\flushright -- Aaron\footnote{Aaron (\texttt{@aarontaycc}, \texttt{@fiatminimalist}), tweet from Dec.
12, 2018~\cite{aarontaycc-tweet}}
\end{quotation}\end{samepage}

\begin{samepage}\begin{quotation}
\enquote{I've learned more about finance, economics, technology, cryptography, human
psychology, politics, game theory, legislation, and myself in the last three
months of crypto than the last three and a half years of college}
\flushright -- Dunny\footnote{Dunny (\texttt{@BitcoinDunny}), tweet from Nov. 28,
2017~\cite{bitcoindunny-tweet}}
\end{quotation}\end{samepage}

These are just two of the many confessions all over twitter.\footnote{See
\url{http://bit.ly/btc-learned} for more confessions on twitter.} Bitcoin, as
was explored in Lesson \ref{les:1}, is a living thing. Mises argued that
economics also is a living thing. And as we all know from personal experience,
living things are inherently difficult to understand.

\begin{samepage}\begin{quotation}
\enquote{A scientific system is but one station in an endlessly progressing
search for knowledge. It is necessarily affected by the insufficiency
inherent in every human effort. But to acknowledge these facts does
not mean that present-day economics is backward. It merely means that
economics is a living thing --- and to live implies both imperfection
and change.}
\flushright -- Ludwig von Mises\footnote{Ludwig von Mises, \textit{Human Action}
\cite{human-action}}
\end{quotation}\end{samepage}

We all read about various financial crises in the news, wonder about how
these big bailouts work and are puzzled over the fact that no one ever
seems to be held accountable for damages which are in the trillions. I
am still puzzled, but at least I am starting to get a glimpse of what is
going on in the world of finance.

Some people even go as far as to attribute the general ignorance on
these topics to systemic, willful ignorance. While history, physics,
biology, math, and languages are all part of our education, the world of
money and finance surprisingly is only explored superficially, if at
all. I wonder if people would still be willing to accrue as much debt as
they currently do if everyone would be educated in personal finance and
the workings of money and debt. Then I wonder how many layers of
aluminum make an effective tinfoil hat. Probably three.

\begin{samepage}\begin{quotation}
\enquote{Those crashes, these bailouts, are not accidents. And neither is it an
accident that there is no financial education in school. [...] It's
premeditated. Just as prior to the Civil War it was illegal to educate a slave,
we are not allowed to learn about money in school.}
\flushright -- Robert Kiyosaki\footnote{Robert Kiyosaki, \textit{Why the Rich
are Getting Richer}\cite{robert-kiyosaki}}
\end{quotation}\end{samepage}

Like in The Wizard of Oz, we are told to pay no attention to the man behind the
curtain. Unlike in The Wizard of Oz, we now have real
wizardry\footnote{\url{http://bit.ly/btc-wizardry}}: a censorship-resistant,
open, borderless network of value-transfer. There is no curtain, and the magic
is visible to anyone.\footnote{\url{https://github.com/bitcoin/bitcoin}}

\paragraph{Bitcoin taught me to look behind the curtain and face my financial
ignorance.}

% ---
%
% #### Down the Rabbit Hole
%
% - [Human Action][Ludwig von Mises] by Ludwig von Mises
% - [Why the Rich are Getting Richer][Robert Kiyosaki] by Robert Kiyosaki
%
% [real wizardry]: https://external-preview.redd.it/8d03MWWOf2HIyKrT8ThBGO4WFv-u25JaYqhbEO9b1Sk.jpg?width=683&auto=webp&s=dc5922d84717c6a94527bafc0189fd4ca02a24bb
% [visible to anyone]: https://github.com/bitcoin/bitcoin
%
% <!-- Wikipedia -->
% [alice]: https://en.wikipedia.org/wiki/Alice%27s_Adventures_in_Wonderland
% [carroll]: https://en.wikipedia.org/wiki/Lewis_Carroll
