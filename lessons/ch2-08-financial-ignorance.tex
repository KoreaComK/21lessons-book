\chapter{Ignorância financeira}
\label{les:8}

\begin{chapquote}{Lewis Carroll, \textit{Alice no País das Maravilhas}}
\enquote{Ela iria pensar que eu sou uma garotinha ignorante por perguntar! Não, não vou perguntar nunca. Talvez eu possa ver o nome escrito em algum lugar..}
\end{chapquote}

Uma das coisas mais surpreendentes para mim foi a quantidade de finanças, economia e psicologia necessária para obter uma compreensão do que, à primeira vista, parece ser um sistema puramente \textit{técnico} --- uma rede de computadores. Parafraseando um baixinho com pés peludos: \enquote{É um negócio perigoso, Frodo, pisar no Bitcoin. Você lê o whitepaper e, se não se controlar, não há como saber para onde pode ser levado.}

Para entender um novo sistema monetário você precisa se familiarizar com o antigo. Comecei a perceber logo que a quantidade de educação financeira que desfrutei no sistema educacional foi basicamente \textit{zero}.

\paragraph{}
Como uma criança de cinco anos, comecei a me fazer muitas perguntas. Como funciona o sistema bancário? Como funciona o mercado de ações? O que é moeda fiduciária? O que é dinheiro \textit{normal}? Por que existe tanta dívida? \footnote{\url{https://www.usdebtclock.org/}} Quanto dinheiro é impresso e quem decide isso?

\newpage

Depois de um leve pânico sobre a extensão da minha ignorância, encontrei segurança ao perceber que estava em boa companhia.

\begin{quotation}\begin{samepage}
\enquote{Não é irônico que o Bitcoin tenha me ensinado mais sobre dinheiro do que todos esses anos que passei trabalhando para instituições financeiras\ldots incluindo a minha carreira em um banco central?}
\begin{flushright} -- Aaron\footnote{Aaron (\texttt{@aarontaycc}, \texttt{@fiatminimalist}), tweet de 12 de dezembro de 2018~\cite{aarontaycc-tweet}}
\end{flushright}\end{samepage}\end{quotation}

\begin{quotation}\begin{samepage}
\enquote{Aprendi mais sobre finanças, economia, tecnologia, criptografia, psicologia humana, política, teoria dos jogos, legislação e sobre mim mesmo nos últimos três meses de cripto do que nos últimos três anos e meio de faculdade.}
\begin{flushright} -- Dunny\footnote{Dunny (\texttt{@BitcoinDunny}), tweet de 28 de Novembro de 2017~\cite{bitcoindunny-tweet}}
\end{flushright}\end{samepage}\end{quotation}

Estas são apenas duas das muitas confissões em todo o Twitter. \footnote{Veja \url{http://bit.ly/btc-learned} para mais confissões no Twitter.} O Bitcoin, como foi explorado na Lição \ref{les:1}, é uma coisa viva. Mises argumentou que a economia também é uma coisa viva. E, como todos sabemos por experiência pessoal, as coisas vivas são inerentemente difíceis de entender.

\begin{quotation}\begin{samepage}
\enquote{Um sistema científico é apenas uma estação em uma busca interminável pelo conhecimento. É necessariamente afetado pela insuficiência inerente a todo esforço humano. Mas reconhecer esses fatos não significa que a economia atual esteja defasada. Significa apenas que a economia é uma coisa viva --- e viver implica imperfeição e mudança.}
\begin{flushright} -- Ludwig von Mises\footnote{Ludwig von Mises, \textit{Ação Humana}
\cite{human-action}}
\end{flushright}\end{samepage}\end{quotation}

Todos nós lemos sobre várias crises financeiras no noticiário, nos perguntamos como funcionam esses grandes resgates e ficamos intrigados com o fato de que ninguém parece jamais ser responsabilizado pelos danos, que estão na casa dos trilhões. Ainda estou confuso, mas pelo menos estou começando a ter um vislumbre do que está acontecendo no mundo das finanças.

Algumas pessoas chegam ao ponto de atribuir a ignorância geral sobre esses tópicos à ignorância intencional e sistêmica. Embora história, física, biologia, matemática e línguas façam parte de nossa educação, o mundo do dinheiro e das finanças, surpreendentemente, só é explorado superficialmente, se é que é explorado. Eu me pergunto se as pessoas ainda estariam dispostas a acumular tantas dívidas como fazem atualmente se todos fossem educados em finanças pessoais e no funcionamento do dinheiro e das dívidas. Então eu me pergunto: quantas camadas de alumínio formam um chapéu de papel alumínio eficaz? Provavelmente três.

\begin{quotation}\begin{samepage}
\enquote{Essas crises, esses resgates, não são acidentes. E também não é por acaso que não há educação financeira na escola. [...] É premeditado. Assim como antes da Guerra Civil era ilegal educar um escravo, não podemos aprender sobre o dinheiro na escola.}
\begin{flushright} -- Robert Kiyosaki\footnote{Robert Kiyosaki, \textit{Por Que o Rico Está Ficando Mais Rico}\cite{robert-kiyosaki}}
\end{flushright}\end{samepage}\end{quotation}

Como na história de O mágico de Oz, somos orientados a não dar atenção ao homem por trás da cortina. Ao contrário de O Mágico de Oz, agora temos magia real \footnote{\url{http://bit.ly/btc-wizardry}}: uma rede de transferência de valor aberta, resistente à censura e sem fronteiras. Não há cortina, e a magia é visível para qualquer um. \footnote{\url{https://github.com/bitcoin/bitcoin}}

\paragraph{Bitcoin me ensinou a olhar atrás da cortina e enfrentar minha ignorância financeira.}

% ---
%
% #### Down the Rabbit Hole
%
% - [Human Action][Ludwig von Mises] by Ludwig von Mises
% - [Why the Rich are Getting Richer][Robert Kiyosaki] by Robert Kiyosaki
%
% [real wizardry]: https://external-preview.redd.it/8d03MWWOf2HIyKrT8ThBGO4WFv-u25JaYqhbEO9b1Sk.jpg?width=683&auto=webp&s=dc5922d84717c6a94527bafc0189fd4ca02a24bb
% [visible to anyone]: https://github.com/bitcoin/bitcoin
%
% <!-- Wikipedia -->
% [alice]: https://en.wikipedia.org/wiki/Alice%27s_Adventures_in_Wonderland
% [carroll]: https://en.wikipedia.org/wiki/Lewis_Carroll
