\chapter{Dinheiro}
\label{les:11}

\begin{chapquote}{O Sábio - Alice no País das Maravilhas}
\enquote{“Quando jovem”, o sábio de cãs disse no ato,\\
“manteve-me lépido e forte este unguento\\
— não queres comprá-lo? \\
É barato: eu vendo a um tostão cada pote.”}
\end{chapquote}

O que é o dinheiro? Nós o usamos todos os dias, mas essa pergunta é surpreendentemente difícil de ser respondida. Dependemos dele em grandes e pequenas quantidades e se tivermos muito pouco nossas vidas se tornarão muito difíceis. No entanto, raramente pensamos sobre o que supostamente faz o mundo girar. O Bitcoin me forçou a responder a esta pergunta repetidamente: Mas o que diabos é o dinheiro?

Em nosso mundo \enquote{moderno}, a maioria das pessoas provavelmente pensará em pedaços de papel quando falam sobre dinheiro, embora a maior parte do nosso dinheiro seja apenas um número na conta bancária. Já estamos usando zeros e uns como dinheiro, então como o Bitcoin é diferente? O Bitcoin é diferente por, em essência, ser um \textit{tipo} de dinheiro muito diferente do que usamos atualmente. Para entender isso, teremos que examinar mais de perto o que é dinheiro, como surgiu e por que o ouro e a prata foram usados na maior parte da história comercial.

\paragraph{}
Conchas do mar, ouro, prata, papel, bitcoin. No final, \textbf{dinheiro é tudo o que as pessoas decidem usar como tal}, não importando forma ou a falta dela.

O dinheiro, como invenção, é engenhoso. Um mundo sem dinheiro é extremamente complicado. Quantos peixes irão me comprar um par de sapatos novos? De quantas vacas vou precisar para comprar uma casa? E se eu não precisar de nada agora, mas precisar me livrar das minhas maçãs que irão estragar? Você não precisa de muita imaginação para perceber que uma economia de escambo é irritantemente ineficiente.

A melhor coisa sobre dinheiro é que ele pode ser trocado por \textit{qualquer outra coisa} --- essa é uma grande invenção! Como Nick Szabo \footnote{\url{http://unenumerated.blogspot.com/}} resumiu brilhantemente no seu texto \textit{Shelling Out: The Origins of Money}\cite{shelling-out}, nós, seres humanos, usamos todos os tipos de coisas como dinheiro: miçangas feitas de materiais raros como marfim, conchas ou ossos especiais, vários tipos de joias e, mais tarde, metais raros como prata e ouro.

\begin{quotation}\begin{samepage}
\enquote{Nesse sentido, é mais típico de um metal precioso. Ao invés de
mudar o suprimento para manter o mesmo valor, a quantidade é
predeterminada e o valor é que muda.}
\begin{flushright} -- Satoshi Nakamoto\footnote{Satoshi Nakamoto, em resposta a Sepp Hasslberger \cite{satoshi-precious-metal}}
\end{flushright}\end{samepage}\end{quotation}

Sendo criaturas preguiçosas que somos, não pensamos muito em como as coisas funcionam. Dinheiro, para a maioria de nós, funciona muito bem. Como acontece com nossos carros ou computadores, a maioria de nós só é forçada a pensar sobre o funcionamento interno dessas coisas caso elas quebrem. As pessoas que viram suas economias desaparecer por causa da hiperinflação sabem o valor de um bom dinheiro, assim como as pessoas que viram seus amigos e familiares desaparecerem por causa das atrocidades da Alemanha nazista ou da Rússia soviética sabem o valor da privacidade.

O problema com o dinheiro é que ele abrange tudo o que existe. O dinheiro é a metade de cada transação, o que confere enorme poder aos responsáveis pela sua criação.

\begin{quotation}\begin{samepage}
\enquote{Dado que o dinheiro é a metade de cada transação comercial e que civilizações inteiras literalmente ascendem e são destruídas com base na qualidade do seu dinheiro, estamos falando de um poder incrível, que voa na calada da noite. É o poder de tecer ilusões que parecem reais enquanto duram. Esse é o cerne do poder do Fed.}
\begin{flushright} -- Ron Paul\footnote{Ron Paul, \textit{O Fim do Fed} \cite{end-the-fed}}
\end{flushright}\end{samepage}\end{quotation}

O Bitcoin remove pacificamente esse poder, uma vez que ele elimina a criação de dinheiro, e faz isso sem o uso da força.

O dinheiro passou por várias iterações. A maioria das delas foi boa. Elas melhoraram nosso dinheiro de um jeito ou de outro. Muito recentemente, porém, o funcionamento interno do nosso dinheiro foi corrompido. Hoje, quase todo o nosso dinheiro é simplesmente criado \textit{do nada} por quem tem poder para tal. Para entender como isso aconteceu, tive de aprender sobre a história e a subsequente queda do dinheiro.

Resta saber se será necessária uma série de catástrofes ou simplesmente um esforço educacional monumental para corrigir essa corrupção. Rezo aos deuses do dinheiro forte para que seja a segunda opção.

\paragraph{O Bitcoin me ensinou o que é dinheiro.}

% ---
%
% #### Down the Rabbit Hole
%
% - [End the Fed][Ron Paul] by Ron Paul
% - [Money, blockchains, and social scalability][social-scalability] by Nick Szabo
%
% [social-scalability]: https://unenumerated.blogspot.co.at/2017/02/money-blockchains-and-social-scalability.html
%
