\chapter{Replicação e Localidade}
\label{les:3}

\begin{chapquote}{Lewis Carroll, \textit{Alice no País das Maravilhas}}
Em seguida uma voz irada, do Coelho: \enquote{Pat, Pat! onde você está?}
\end{chapquote}

Deixando a mecânica quântica de lado, a localidade não é um problema no mundo físico. A questão \textit{\enquote{Onde está X?}} Pode ser respondida de forma significativa, não importa se X é uma pessoa ou um objeto. No mundo digital, a pergunta do \textit{onde} já é mais complicada, mas não impossível de responder. Onde estão seus e-mails realmente? Uma resposta não muito boa seria \enquote{na nuvem} que nada mais é que o computador de outra pessoa. Ainda assim, se você quisesse rastrear cada dispositivo de armazenamento que contém seus e-mails, você poderia, em teoria, localizá-los.

Com o bitcoin, a pergunta do \enquote{onde} é \textit{realmente} complicada. Onde, exatamente, estão seus bitcoins?

\begin{quotation}\begin{samepage}
\enquote{Abri os olhos, olhei em volta e fiz a pergunta inevitável, tradicional e lamentavelmente banal do pós-operatório: "Onde estou?"}
\begin{flushright} -- Daniel Dennett\footnote{Daniel Dennett, \textit{Where Am I?}~\cite{where-am-i}}
\end{flushright}\end{samepage}\end{quotation}

O problema é duplo. Primeiro, o livro razão distribuído é distribuído por replicação completa, o que significa que o livro razão está em toda parte. Em segundo lugar, não existem bitcoins. Não apenas fisicamente, mas \textit{tecnicamente}.

O Bitcoin rastreia um conjunto de saídas de transações não gastas, sem nunca ter que se referir a uma entidade que represente um bitcoin. A existência de um bitcoin é baseada observando-se o conjunto de saídas de transações não gastas e chamando cada entrada com 100 milhões de unidades básicas de bitcoin.

\begin{quotation}\begin{samepage}
\enquote{Onde está, neste momento, em trânsito? [...] primeiro, não há bitcoins. Simplesmente não existem. Eles não existem. Existem entradas em um livro razão que é compartilhado [...] Eles não existem em nenhum local físico. O livro razão existe em todos os locais físicos, essencialmente. A geografia não faz sentido neste mundo --- não vai ajudá-lo a descobrir a sua política aqui.}
\begin{flushright} -- Peter Van Valkenburgh\footnote{Peter Van Valkenburgh on the \textit{What Bitcoin Did} podcast, episode 49 \cite{wbd049}}
\end{flushright}\end{samepage}\end{quotation}

Então, o que você realmente possui quando diz \textit{\enquote{Eu tenho um bitcoin}} se não há bitcoins? Bem, lembra de todas essas palavras estranhas que você foi forçado a escrever quando usou uma carteira? Acontece que essas palavras mágicas são o que você realmente possui: um feitiço mágico \footnote{The Magic Dust of Cryptography: Como a informação digital está mudando nossa sociedade \cite{gigi:magic-spell}} que pode ser usado para adicionar algumas entradas ao livro razão público --- as chaves para \enquote{mover} alguns bitcoins. É por isso que, para todos os efeitos, suas chaves privadas \textit{são} seus bitcoins. Se você acha que estou inventando tudo isso, sinta-se à vontade para me enviar suas chaves privadas.

\paragraph{O Bitcoin me ensinou que localidade é um negócio complicado.}

% ---
%
% #### Through the Looking-Glass
%
% - [The Magic Dust of Cryptography: How digital information is changing our society][a magic spell]
%
% #### Down the Rabbit Hole
%
% - [Where Am I?][Daniel Dennett] by Daniel Dennett
% - 🎧 [Peter Van Valkenburg on Preserving the Freedom to Innovate with Public Blockchains][wbd049] WBD #49 hosted by Peter McCormack
%
% <!-- Through the Looking-Glass -->
% [a magic spell]: 
%
% <!-- Down the Rabbit Hole -->
% [Daniel Dennett]: https://www.lehigh.edu/~mhb0/Dennett-WhereAmI.pdf
% [1st Amendment]: https://en.wikipedia.org/wiki/First_Amendment_to_the_United_States_Constitution
% [wbd049]: https://www.whatbitcoindid.com/podcast/coin-centers-peter-van-valkenburg-on-preserving-the-freedom-to-innovate-with-public-blockchains
%
% <!-- Wikipedia -->
% [alice]: https://en.wikipedia.org/wiki/Alice%27s_Adventures_in_Wonderland
% [carroll]: https://en.wikipedia.org/wiki/Lewis_Carroll
