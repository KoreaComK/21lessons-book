
\chapter{A escassez da escassez}
\label{les:2}

\begin{chapquote}{Alice}
\enquote{É o suficiente\ldots eu espero não crescer mais\ldots}
\end{chapquote}

Em geral, o avanço da tecnologia parece tornar as coisas mais abundantes. Cada vez mais pessoas podem desfrutar do que antes eram bens luxuosos. Em breve, todos nós viveremos como reis. A maioria de nós já vive assim. Como Peter Diamandis escreveu em Abundance~\cite{abundance}: \enquote{A tecnologia é um mecanismo de liberação de recursos. Pode tornar o que antes era escasso em abundante.}

O Bitcoin, uma tecnologia avançada em si, quebra essa tendência e cria uma nova commodity que é realmente escassa. Alguns até argumentam que é uma das coisas mais raras do universo. A oferta não pode ser inflacionada, não importa quanto esforço seja despendido para se criar mais.

\begin{quotation}\begin{samepage}
\enquote{Apenas duas coisas são genuinamente escassas: O Tempo e o Bitcoin.}
\begin{flushright} -- Saifedean Ammous\footnote{Apresentação do livro The Bitcoin Standard~\cite{bitcoinstandard-pres}}
\end{flushright}\end{samepage}\end{quotation}

Paradoxalmente, ele o faz por meio de um mecanismo de cópia. As transações são transmitidas, os blocos são propagados, o livro razão distribuído é --- bem, você adivinhou --- distribuído. Todas essas são apenas palavras bonitas para dizer a mesma coisa: copiar. Caramba, o Bitcoin até mesmo se copia em quantos computadores puder, incentivando pessoas individuais a executar nodes completos e a minerar novos blocos.

Toda essa duplicação funciona maravilhosamente em conjunto em um esforço concentrado para produzir escassez.

\paragraph{Em tempos de abundância, o Bitcoin me ensinou o que é a verdadeira escassez.}

% ---
%
% #### Through the Looking-Glass
%
% - [Lesson 14: Sound money][lesson14]
%
% #### Down the Rabbit Hole
%
% - [The Bitcoin Standard: The Decentralized Alternative to Central Banking][bitcoin-standard]
% - [Abundance: The Future Is Better Than You Think][Abundance] by Peter Diamandis
% - [Presentation on The Bitcoin Standard][bitcoin-standard-presentation] by Saifedean Ammous
% - [Modeling Bitcoin's Value with Scarcity][planb-scarcity] by PlanB
% - 🎧 [Misir Mahmudov on the Scarcity of Time & Bitcoin][tftc60] TFTC #60 hosted by Marty Bent
% - 🎧 [PlanB – Modelling Bitcoin's digital scarcity through stock-to-flow techniques][slp67] SLP #67 hosted by Stephan Livera
%
% <!-- Through the Looking-Glass -->
% [lesson14]: {{ 'bitcoin/lessons/ch2-14-sound-money' | absolute_url }}
%
% <!-- Down the Rabbit Hole -->
% [Abundance]: https://www.diamandis.com/abundance
% [bitcoin-standard]: http://amzn.to/2L95bJW
% [bitcoin-standard-presentation]: https://www.bayernlb.de/internet/media/de/ir/downloads_1/bayernlb_research/sonderpublikationen_1/bitcoin_munich_may_28.pdf
% [planb-scarcity]: https://medium.com/@100trillionUSD/modeling-bitcoins-value-with-scarcity-91fa0fc03e25
% [tftc60]: https://anchor.fm/tales-from-the-crypt/episodes/Tales-from-the-Crypt-60-Misir-Mahmudov-e3aibh
% [slp67]: https://stephanlivera.com/episode/67
%
% <!-- Wikipedia -->
% [alice]: https://en.wikipedia.org/wiki/Alice%27s_Adventures_in_Wonderland
% [carroll]: https://en.wikipedia.org/wiki/Lewis_Carroll
