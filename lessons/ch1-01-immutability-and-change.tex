\chapter{Imutabilidade e mudança}
\label{les:1}

\begin{chapquote}{Alice}
\enquote{O que será que mudou à noite? Deixe-me ver: eu era a mesma quando acordei de manhã? Tenho a impressão de ter me sentido um pouco diferente. Mas se eu não sou a mesma, a próxima questão é “Quem sou eu?” Ah! esta é a grande confusão!}
\end{chapquote}

O Bitcoin é por padrão difícil de se descrever. É uma \textit {coisa nova}, e qualquer tentativa de fazer uma comparação com conceitos anteriores - seja chamando-o de ouro digital ou de dinheiro da internet  - está fadada a ficar aquém do todo. Qualquer que seja sua analogia favorita, dois aspectos do Bitcoin são absolutamente essenciais: descentralização e imutabilidade.

\paragraph{}
Uma maneira de pensar sobre o Bitcoin é como um contrato social automatizado \footnote{Hasu, Unpacking Bitcoin's Social Contract~\cite {social-contract}}. O software é apenas uma peça do quebra-cabeça, e esperar mudar o Bitcoin mudando o software é um exercício de futilidade. Seria preciso convencer o restante da rede a adotar as mudanças, o que é mais um esforço psicológico do que de engenharia de software.

\paragraph{}
O que se segue pode parecer absurdo à primeira vista, como tantas outras coisas neste espaço, mas acredito que seja profundamente verdadeiro, no entanto: você não mudará o Bitcoin, mas Bitcoin irá mudar você.

\begin{quotation}\begin{samepage}
\enquote{O Bitcoin irá nos mudar mais do que nós podemos mudá-lo.}
\begin{flushright} -- Marty Bent\footnote{Tales From the Crypt~\cite{tftc21}}
\end{flushright}\end{samepage}\end{quotation}

Levei muito tempo para perceber a profundidade disso. Como o Bitcoin é apenas software e tudo é de código aberto, você pode simplesmente mudar as coisas à vontade, certo? Errado. \textit {Muito} errado. Sem nenhuma surpresa, o criador do Bitcoin sabia disso muito bem.

\begin{quotation}\begin{samepage}
\enquote{A natureza do Bitcoin é tal que no momento que a versão 0.1 foi lançada, o design do núcleo foi definido para o resto da vida.}
\begin{flushright} -- Satoshi Nakamoto\footnote{Postagem no fórum do BitcoinTalk: `Resposta: Transações e Scripts \ldots'~\cite{satoshi-set-in-stone}}
\end{flushright}\end{samepage}\end{quotation}

Muitas pessoas tentaram mudar a natureza do Bitcoin. Até agora, todos falharam. Embora exista um mar infinito de forks e altcoins, a rede Bitcoin ainda faz seu trabalho, assim como fazia quando o primeiro nó estava online. As altcoins não importarão no longo prazo. Os forks acabarão morrendo definhados. O Bitcoin é o que importa. Enquanto nosso entendimento fundamental de matemática e/ou física não mudar, o honeybadger do Bitcoin continuará a não se importar.


\begin{quotation}\begin{samepage}
\enquote{O Bitcoin é o primeiro exemplo de uma nova aforma de vida. Ela vive e respira na internet. Ela vive porque ela pode pagar as pessoas para que ela continue viva. [\ldots] Não pode ser mudada. Não podemos discutir com ela. Não pode ser adulterada. Não pode ser corrompida. Não pode ser parada. [\ldots] Se uma guerra nuclear destruir metade do nosso planeta, ela continuará viva e incorruptível.}
\begin{flushright} -- Ralph Merkle\footnote{DAOs, Democracy and
Governance,~\cite{merkle-dao}}
\end{flushright}\end{samepage}\end{quotation}

O batimento cardíaco da rede Bitcoin durará mais do que todos os nossos.

~

Depois que entendi a citação acima, ela me mudou muito mais do que os blocos anteriores do blockchain do Bitcoin jamais farão. Mudou minha preferência temporal, meu entendimento de economia, minhas visões políticas e muito mais. Maldição, está até mudando a dieta das pessoas \footnote{Inside the World of the Bitcoin
Carnivores,~\cite{carnivores}}. Se tudo isso parece loucura para você, não se preocupe, você está em boa companhia. Tudo isso é uma loucura e, no entanto, está acontecendo.

~

\paragraph{O Bitcoin me ensinou que ele não irá mudar. Eu irei.}

% ---
%
% #### Through the Looking-Glass
%
% - [Bitcoin's Gravity: How idea-value feedback loops are pulling people in][gravity]
% - [Lesson 18: Move slowly and don't break things][lesson18]
%
% #### Down the Rabbit Hole
%
% - [Unpacking Bitcoin's Social Contract][automated social contract]: A framework for skeptics by Hasu
% - [DAOs, Democracy and Governance][Ralph Merkle] by Ralph C. Merkle
% - [Marty's Bent][bent]: A daily newsletter highlighting signal in Bitcoin by Marty Bent
% - [Technical Discussion on Bitcoin's Transactions and Scripts][Satoshi Nakamoto] by Satoshi Nakamoto, Gavin Andresen, and others
% - [Inside the World of the Bitcoin Carnivores][carnivores]: Why a small community of Bitcoin users is eating meat exclusively by Jordan Pearson
% - [Tales From the Crypt][tftc] hosted by Marty Bent
%
% <!-- Internal -->
% [gravity]: 
% [lesson18]: {{ 'bitcoin/lessons/ch3-18-move-slowly-and-dont-break-things' | absolute_url }}
%
% <!-- Further Reading -->
% [automated social contract]: https://medium.com/@hasufly/bitcoins-social-contract-1f8b05ee24a9
% [carnivores]: https://motherboard.vice.com/en_us/article/ne74nw/inside-the-world-of-the-bitcoin-carnivores
% [tftc]: https://tftc.io/tales-from-the-crypt/
% [bent]: https://tftc.io/martys-bent/
%
% <!-- Quotes -->
% [Ralph Merkle]: http://merkle.com/papers/DAOdemocracyDraft.pdf
% [Satoshi Nakamoto]: https://bitcointalk.org/index.php?topic=195.msg1611#msg1611
%
% <!-- Twitter People -->
% [Marty Bent]: https://twitter.com/martybent
%
% <!-- Wikipedia -->
% [alice]: https://en.wikipedia.org/wiki/Alice%27s_Adventures_in_Wonderland
% [carroll]: https://en.wikipedia.org/wiki/Lewis_Carroll