\chapter{Immutability and Change}
\label{les:1}

\begin{chapquote}{Alice}
\enquote{I wonder if I've been changed in the night. Let me think. Was I the same when I
got up this morning? I almost think I can remember feeling a little different.
But if I'm not the same, the next question is `Who in the world am I?' Ah,
that's the great puzzle!}
\end{chapquote}

Bitcoin is inherently hard to describe. It is a \textit{new thing}, and any
attempt to draw a comparison to previous concepts -- be it by calling
it digital gold or the internet of money -- is bound to fall short of
the whole. Whatever your favorite analogy might be, two aspects of
Bitcoin are absolutely essential: decentralization and immutability.

One way to think about Bitcoin is as an automated social contract\footnote{Hasu,
Unpacking Bitcoin's Social Contract~\cite{social-contract}}. The software is
just one piece of the puzzle, and hoping to change Bitcoin by changing the
software is an exercise in futility. One would have to convince the rest of the
network to adopt the changes, which is more a psychological effort than a
software engineering one.

The following might sound absurd at first, like so many other things in
this space, but I believe that it is profoundly true nonetheless: You
won't change Bitcoin, but Bitcoin will change you.

\begin{quotation}
\enquote{Bitcoin will change us more than we will change it.}
\flushright -- Marty Bent\footnote{Tales From the Crypt~\cite{tftc21}}
\end{quotation}

It took me a long time to realize the profundity of this. Since Bitcoin
is just software and all of it is open-source, you can simply change
things at will, right? Wrong. \textit{Very} wrong. Unsurprisingly, Bitcoin's
creator knew this all too well.

\begin{quotation}
\enquote{The nature of Bitcoin is such that once version 0.1 was released, the core
design was set in stone for the rest of its lifetime.}
\flushright -- Satoshi Nakamoto\footnote{BitcoinTalk forum post: `Re:
Transactions and Scripts...'~\cite{satoshi-set-in-stone}}
\end{quotation}

Many people have attempted to change Bitcoin's nature. So far all of
them have failed. While there is an endless sea of forks and altcoins,
the Bitcoin network still does its thing, just as it did when the first
node went online. The altcoins won't matter in the long run. The forks
will eventually starve to death. Bitcoin is what matters. As long as our
fundamental understanding of mathematics and/or physics doesn't change,
the Bitcoin honeybadger will continue to not care.

\begin{quotation}
\enquote{Bitcoin is the first example of a new form of life. It lives and
breathes on the internet. It lives because it can pay people to keep
it alive. ... It can't be changed. It can't be argued with. It
can't be tampered with. It can't be corrupted. It can't be stopped.
... If nuclear war destroyed half of our planet, it would continue
to live, uncorrupted.}
\flushright -- Ralph Merkle\footnote{DAOs, Democracy and
Governance,~\cite{merkle-dao}}
\end{quotation}

The heartbeat of the Bitcoin network will outlast all of ours.

~

Realizing the above changed me way more than the past blocks of the Bitcoin
blockchain ever will. It changed my time preference, my understanding of
economics, my political views, and so much more. Hell, it is even changing
people's diets\footnote{Inside the World of the Bitcoin
Carnivores,~\cite{carnivores}}. If all of this sounds crazy to you, you're in
good company. All of this is crazy, and yet it is happening.

~

\paragraph{Bitcoin taught me that it won't change. I will.}

% ---
%
% #### Through the Looking-Glass
%
% - [Bitcoin's Gravity: How idea-value feedback loops are pulling people in][gravity]
% - [Lesson 18: Move slowly and don't break things][lesson18]
%
% #### Down the Rabbit Hole
%
% - [Unpacking Bitcoin's Social Contract][automated social contract]: A framework for skeptics by Hasu
% - [DAOs, Democracy and Governance][Ralph Merkle] by Ralph C. Merkle
% - [Marty's Bent][bent]: A daily newsletter highlighting signal in Bitcoin by Marty Bent
% - [Technical Discussion on Bitcoin's Transactions and Scripts][Satoshi Nakamoto] by Satoshi Nakamoto, Gavin Andresen, and others
% - [Inside the World of the Bitcoin Carnivores][carnivores]: Why a small community of Bitcoin users is eating meat exclusively by Jordan Pearson
% - [Tales From the Crypt][tftc] hosted by Marty Bent
%
% <!-- Internal -->
% [gravity]: 
% [lesson18]: {{ 'bitcoin/lessons/ch3-18-move-slowly-and-dont-break-things' | absolute_url }}
%
% <!-- Further Reading -->
% [automated social contract]: https://medium.com/@hasufly/bitcoins-social-contract-1f8b05ee24a9
% [carnivores]: https://motherboard.vice.com/en_us/article/ne74nw/inside-the-world-of-the-bitcoin-carnivores
% [tftc]: https://tftc.io/tales-from-the-crypt/
% [bent]: https://tftc.io/martys-bent/
%
% <!-- Quotes -->
% [Ralph Merkle]: http://merkle.com/papers/DAOdemocracyDraft.pdf
% [Satoshi Nakamoto]: https://bitcointalk.org/index.php?topic=195.msg1611#msg1611
%
% <!-- Twitter People -->
% [Marty Bent]: https://twitter.com/martybent
%
% <!-- Wikipedia -->
% [alice]: https://en.wikipedia.org/wiki/Alice%27s_Adventures_in_Wonderland
% [carroll]: https://en.wikipedia.org/wiki/Lewis_Carroll
