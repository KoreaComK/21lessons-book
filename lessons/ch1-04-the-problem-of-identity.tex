\chapter{O problema da Identidade}
\label{les:4}

\begin{chapquote}{Lewis Carroll, \textit{Alice no País das Maravilhas}}
  \enquote{Quem é você?}, perguntou a Lagarta..
\end{chapquote}

Nic Carter, em homenagem ao trabalho de Thomas Nagel em relação a mesma questão relacionada ao morcego, escreveu um excelente artigo que discute a seguinte questão: Como é ser um bitcoin? Ele mostra brilhantemente que blockchains públicas e abertas em geral, e Bitcoin em particular, sofrem do mesmo enigma do navio de Teseu \footnote{Na metafísica da identidade, o navio de Teseu é um experimento mental que levanta a questão de saber se um objeto que teve todos os seus componentes substituídos permanece fundamentalmente o mesmo. ~\cite{wiki: theseus}}: Qual Bitcoin é o Bitcoin verdadeiro?

\begin{quotation}\begin{samepage}
\enquote{Considere quão pouca persistência os componentes do Bitcoin possuem. O código base inteiro foi retrabalhado, alterado e expandido de tal forma que mal se parece com sua versão original. [...] O registro de quem possui o que, o próprio livro razão, é praticamente o único traço persistente da rede [...] Para ser considerado verdadeiramente sem um líder, você deve renunciar à solução fácil de ter uma entidade que pode designar uma chain como sendo a legítima.}
\begin{flushright} -- Nic Carter\footnote{Nic Carter, \textit{Como é ser um bitcoin?} \cite{bitcoin-identity}}
\end{flushright}\end{samepage}\end{quotation}

Parece que o avanço da tecnologia continua nos forçando a levar essas questões filosóficas a sério. Mais cedo ou mais tarde, os carros que dirigem sozinhos serão confrontados com versões reais do dilema do bonde, forçando-os a tomar decisões éticas sobre quais vidas são mais importantes do que outras.

As criptomoedas, especialmente desde o primeiro hard fork, nos forçam a pensar e a concordar sobre a metafísica da identidade. Curiosamente, os dois maiores exemplos que temos até agora levaram a duas respostas diferentes. No dia 1º de agosto de 2017, o Bitcoin se dividiu em dois. O mercado decidiu que a chain inalterada é o Bitcoin original. Um ano antes, em 25 de outubro de 2016, o Ethereum se dividiu em dois. O mercado decidiu que a chain \textit{alterada} é o Ethereum original.

Se devidamente descentralizadas, as questões colocadas pelo paradoxo do \textit{Návio de Teseu}, terão que ser respondidas perpetuamente enquanto essas redes de transferência de valor existirem.

\paragraph{O Bitcoin me ensinou que descentralização contradiz a identidade.}

% ---
%
% #### Down the Rabbit Hole
%
% - [What Is It Like to be a Bat?][in regards to a bat] by Thomas Nagel
% - [What is it like to be a bitcoin?] by Nic Carter
% - [Ship of Theseus], [trolley problem] on Wikipedia
%
% [in regards to a bat]: https://en.wikipedia.org/wiki/What_Is_it_Like_to_Be_a_Bat%3F
% [What is it like to be a bitcoin?]: https://medium.com/s/story/what-is-it-like-to-be-a-bitcoin-56109f3e6753
% [Ship of Theseus]: https://en.wikipedia.org/wiki/Ship_of_Theseus
% [trolley problem]: https://en.wikipedia.org/wiki/Trolley_problem
%
% <!-- Wikipedia -->
% [alice]: https://en.wikipedia.org/wiki/Alice%27s_Adventures_in_Wonderland
% [carroll]: https://en.wikipedia.org/wiki/Lewis_Carroll
