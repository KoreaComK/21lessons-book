\chapter{O Poder da Liberdade de Expressão}
\label{les:6}

\begin{chapquote}{Lewis Carroll, \textit{Alice no País das Maravilhas}}
\enquote{Desculpe-me}, disse o Rato, carrancudo, mas educadamente, \enquote{Você falou alguma coisa?}
\end{chapquote}

O Bitcoin é uma ideia. Uma ideia que, na sua forma atual, é a manifestação de uma máquina movida puramente por texto. Cada aspecto do Bitcoin é um texto: o whitepaper é um texto. O software executado por seus nodes é um texto. O livro razão é um texto. As transações são textos. As chaves públicas e privadas são textos. Cada aspecto do Bitcoin é um texto e, portanto, equivalente à fala.

\begin{quotation}\begin{samepage}
\enquote{Congress shall make no law respecting an establishment of religion,
or prohibiting the free exercise thereof; or abridging the freedom of
speech, or of the press; or the right of the people peaceably to
assemble, and to petition the Government for a redress of grievances.}

\enquote{O Congresso não fará nenhuma lei respeitando o estabelecimento de uma religião, ou proibindo o seu livre exercício; ou irá restringir a liberdade de expressão ou de imprensa; ou o direito do povo de se reunir pacificamente e de peticionar ao governo para a reparação de queixas.}
\begin{flushright} --- Primeira Emenda à Constituição dos Estados Unidos
\end{flushright}\end{samepage}\end{quotation}

Embora a batalha final das Crypto Wars \footnote{The \textit{Crypto Wars} seja um nome não oficial para as tentativas dos EUA e governos aliados de minar a criptografia. ~\cite{eff-cryptowars} ~\cite{wiki:cryptowars}} ainda não foi combatida, será muito difícil criminalizar uma ideia, muito menos uma ideia que se baseia na troca de mensagens de texto. Cada vez que um governo tenta proibir um texto ou discurso, escorregamos no caminho do absurdo que inevitavelmente leva a abominações como números ilegais \footnote{Um número ilegal é um número que representa informações que são ilegais de possuir, proferir, propagar ou de outra forma transmitir em alguma jurisdição legal. \cite{wiki:illegal-number}} e primos ilegais \footnote{Um número primo ilegal é um número primo que representa informação cuja posse ou distribuição é proibida em algumas jurisdições legais. Um dos primeiros primos ilegais foi descoberto em 2001. Quando interpretado de uma maneira particular, ele descreve um programa de computador que ignora o esquema de gerenciamento de direitos digitais usado em DVDs. A distribuição de tal programa nos Estados Unidos é ilegal de acordo com a Lei de Direitos Autorais do Milênio Digital. Um primo ilegal é um tipo de número ilegal. \cite{wiki:illegal-prime}}.

Enquanto houver uma parte do mundo onde a liberdade de expressão seja livre como na verdadeira \textit{liberdade}, O Bitcoin é imparável.

\begin{quotation}\begin{samepage}
\begin{flushright}
\enquote{Não há nenhum ponto em qualquer transação do Bitcoin onde ele deixe de ser \textit{texto}. Ele é \textit{totalmente texto}, o tempo todo. [...] O Bitcoin é \textit{texto}. Bitcoin é um \textit{fala}. Não pode ser regulamentado em um país livre como os EUA, com direitos inalienáveis garantidos e uma Primeira Emenda que exclui explicitamente o ato de publicação da supervisão do governo.}
-- Beautyon\footnote{Beautyon, \textit{Por que os EUA não podem regular o Bitcoin} \cite{america-regulate-bitcoin}}
\end{flushright}\end{samepage}\end{quotation}

\paragraph{O Bitcoin me ensinou que em uma sociedade livre, a liberdade de expressão e o software livre são imparáveis.}

% ---
%
% #### Through the Looking-Glass
%
% - [The Magic Dust of Cryptography: How digital information is changing our society][a magic spell]
%
% #### Down the Rabbit Hole
%
% - [Why America can't regulate Bitcoin][Beautyon] by Beautyon
% - [First Amendment to the United States Constitution][1st Amendment], [Crypto Wars], [illegal numbers], [illegal primes] on Wikipedia
%
% <!-- Through the Looking-Glass -->
% [a magic spell]: 
%
% <!-- Down the Rabbit Hole -->
% [1st Amendment]: https://en.wikipedia.org/wiki/First_Amendment_to_the_United_States_Constitution
% [Crypto Wars]: https://en.wikipedia.org/wiki/Crypto_Wars
% [illegal numbers]: https://en.wikipedia.org/wiki/Illegal_number
% [illegal primes]: https://en.wikipedia.org/wiki/Illegal_prime
% [Beautyon]: https://hackernoon.com/why-america-cant-regulate-bitcoin-8c77cee8d794
%
% <!-- Wikipedia -->
% [alice]: https://en.wikipedia.org/wiki/Alice%27s_Adventures_in_Wonderland
% [carroll]: https://en.wikipedia.org/wiki/Lewis_Carroll
