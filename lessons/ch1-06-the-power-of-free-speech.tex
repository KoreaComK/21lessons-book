\chapter{The Power of Free Speech}
\label{les:6}

\begin{chapquote}{Lewis Carroll, \textit{Alice in Wonderland}}
\enquote{I beg your pardon?} said the mouse, frowning, but very politely, \enquote{did you speak?}
\end{chapquote}

Bitcoin is an idea. An idea which, in its current form, is the
manifestation of a machinery purely powered by text. Every aspect of
Bitcoin is text: The whitepaper is text. The software which is run by
its nodes is text. The ledger is text. Transactions are text. Public and
private keys are text. Every aspect of Bitcoin is text, and thus
equivalent to speech.

\begin{samepage}\begin{quotation}
\enquote{Congress shall make no law respecting an establishment of religion,
or prohibiting the free exercise thereof; or abridging the freedom of
speech, or of the press; or the right of the people peaceably to
assemble, and to petition the Government for a redress of grievances.}
\flushright -- First Amendment to the U.S. Constitution
\end{quotation}\end{samepage}

Although the final battle of the Crypto Wars\footnote{The \textit{Crypto Wars}
is an unofficial name for the U.S. and allied governments' attempts to undermine
encryption.~\cite{eff-cryptowars}~\cite{wiki:cryptowars}} has not been fought
yet, it will be very difficult to criminalize an idea, let alone an idea which
is based on the exchange of text messages. Every time a government tries to
outlaw text or speech, we slip down a path of absurdity which inevitably leads
to abominations like illegal numbers\footnote{An illegal number is a number that
represents information which is illegal to possess, utter, propagate, or
otherwise transmit in some legal jurisdiction.\cite{wiki:illegal-number}} and
illegal primes\footnote{An illegal prime is a prime number that represents
information whose possession or distribution is forbidden in some legal
jurisdictions. One of the first illegal primes was found in 2001. When
interpreted in a particular way, it describes a computer program that bypasses
the digital rights management scheme used on DVDs. Distribution of such a
program in the United States is illegal under the Digital Millennium Copyright
Act. An illegal prime is a kind of illegal number.\cite{wiki:illegal-prime}}.

As long as there is a part of the world where speech is free as in
\textit{freedom}, Bitcoin is unstoppable.

\begin{samepage}\begin{quotation}
\enquote{There is no point in any Bitcoin transaction that Bitcoin ceases to be
\textit{text}. It is \textit{all text}, all the time. [...] Bitcoin is
\textit{text}. Bitcoin is \textit{speech}. It cannot be regulated in a free
country like the USA with guaranteed inalienable rights and a First Amendment
that explicitly excludes the act of publishing from government oversight.}
\flushright -- Beautyon\footnote{Beautyon, \textit{Why America can't regulate
Bitcoin} \cite{america-regulate-bitcoin}}
\end{quotation}\end{samepage}

\paragraph{Bitcoin taught me that in a free society, free speech and free software
are unstoppable.}

% ---
%
% #### Through the Looking-Glass
%
% - [The Magic Dust of Cryptography: How digital information is changing our society][a magic spell]
%
% #### Down the Rabbit Hole
%
% - [Why America can't regulate Bitcoin][Beautyon] by Beautyon
% - [First Amendment to the United States Constitution][1st Amendment], [Crypto Wars], [illegal numbers], [illegal primes] on Wikipedia
%
% <!-- Through the Looking-Glass -->
% [a magic spell]: 
%
% <!-- Down the Rabbit Hole -->
% [1st Amendment]: https://en.wikipedia.org/wiki/First_Amendment_to_the_United_States_Constitution
% [Crypto Wars]: https://en.wikipedia.org/wiki/Crypto_Wars
% [illegal numbers]: https://en.wikipedia.org/wiki/Illegal_number
% [illegal primes]: https://en.wikipedia.org/wiki/Illegal_prime
% [Beautyon]: https://hackernoon.com/why-america-cant-regulate-bitcoin-8c77cee8d794
%
% <!-- Wikipedia -->
% [alice]: https://en.wikipedia.org/wiki/Alice%27s_Adventures_in_Wonderland
% [carroll]: https://en.wikipedia.org/wiki/Lewis_Carroll
